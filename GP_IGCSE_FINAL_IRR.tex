\documentclass[a4paper, 11pt]{article}

\usepackage[protrusion=true,expansion=true]{microtype} % Better typography
\usepackage{graphicx} % Required for including pictures
\usepackage{wrapfig} % Allows in-line images

\usepackage{mathpazo} % Use the Palatino font
\usepackage[T1]{fontenc} % Required for accented characters
\linespread{1.05} % Change line spacing here, Palatino benefits from a slight increase by default

\makeatletter

\renewcommand{\maketitle}{
\begin{flushright}
{\LARGE\@title}

\vspace{20pt}

{\large\@author}
\\\@date % Date

\vspace{40pt} % Some vertical space between the author block and abstract
\end{flushright}
}

\title{\textbf{How does the digital world affect the wellbeing of its users?}}

\author{\textsc{Vishnu Vivek Nittoor} \\{\textit{The International School Bangalore}}}

\date{\today}

\begin{document}
\maketitle

\vspace{30pt}

\section*{Introduction}

The digital world is a culmination of humanity's efforts of communication and connection, which has evolved into a vast realm of possibilities in the past few decades. From dial-up modems to extremely fast fiberoptic cables, the availability, accessibility and increasing speed of the internet has facilitated this evolution. The internet is a tool with which users can communicate with others at staggering speeds, comparable to the speed of light \cite{speedoflight-communication}, something that people nowadays take for granted more often than not. The digital world has witnessed indiscriminate use in the past decades, with now 3.010 billion people having access to the internet \cite{world-population-online}.

For the purposes of this report, 'wellbeing' refers to an individual's state of being comfortable, healthy, or happy. This can also be further divided into physical, mental, and social wellbeing \cite{physical-wellbeing-definition, who-mental-wellbeing, social-wellbeing-definition}. This report will explore the ways in which the digital world affects these users' wellbeing.

\section{National Perspective}

According to this article \cite{teens-social-media-brain}, researchers at UCLA’s brain mapping centre found that when teenagers’ photos get lots of “likes” on social media apps, their brains’ reward centre is stimulated. This can cause social media to act like any other addictive substance, training the brain to use it because of the feeling associated in their brains with getting likes. Adolescents are more susceptible to this. It turns out that according to this study on Indian youth, 73\% of respondents agreed that social media is highly addictive. Just as the same for any addictive substance, this leads to the assumption that social media negatively affects the wellbeing of its users.

Related to this is the advanced algorithms of sites like YouTube. These are engineered in a way that takes into account the past videos watched by an individual, the interest exhibited in them, and uses extremely complex artificial intelligence and machine learning algorithms to determine  what videos the user is likely to click on next. This results in a user being constantly shown content which interests them, keeping them on the site for the longest amount of time so that ad revenue can be generated. This causes them to be addicted to the site.

The same study on the Indian youth shows that 76\% of respondents agreed that social networking leads to restlessness and sleeplessness. This is also confirmed by an article \cite{your-gadgets-are-robbing-you-of-your-sleep}, which says that according to the 2017 survey conducted by Philips, 36\% of Indians blamed technology for poor sleep. This is due to many reasons; one being the connectivity that the digital world provides. Because of addiction to social platforms, people cannot put down their phones because they will constantly be receiving notifications, messages and content from other people. The article says that “Researchers say we get addicted to these sounds over time”. This addiction to the sounds cause people to become alert whenever they get a notification - even when they are almost asleep. Since they have trained their brains to respond to the notification as soon as it arrives, they react to their first instinct and respond to it. Their response prompts the contact who is messaging them to respond due to the same phenomenon - and this continues. This can cause sleepless nights.

Cyber-bullying or cyber harassment is a form of bullying or harassment using electronic means. Cyber-bullying is also known as online bullying. This can cause victims to suffer a considerable drop in self-esteem, while being a constant source of distress in their lives. This causes them to spend time in withdrawal from family members and friends, and experience anxiety regarding other individuals checking their mobile phone, laptop, etc, and can even lead to victims committing suicide in extreme cases. Cyber-bullying extremely negatively affects an individual’s social well-being.

Based on a study conducted on the Indian youth regarding the impacts of social networking, 61\% of respondents agreed that social media gives rise to cyber-bullying. According to a 2014 study conducted by the Internet security company, McAfee, “Half of the youth in India have had some experience with cyber-bullying”. If this is a consequence of social media, then data leads us to believe that Indian users are prone to be very negatively affected by exposure to the digital world.

On the other hand, the Indian government has launched a campaign called 'Digital India', which aims to "empower every citizen with access to digital services, knowledge and information", because "Digital technologies are being increasingly used by us in everyday lives from retail stores to government offices. They help us to connect with each other and also to share information on issues and concerns faced by us. In some cases they also enable resolution of those issues in near real time" \cite{digital-india-gov}. It includes initiatives to reach internet to rural areas and improve accessibility for all Indian citizens. Though many groups of people express skepticism \cite{digital-india-thehindu, digital-india-thehindu-2}, this reflects the Indian government's belief that access to the digital world improves the wellbeing of India's citizens, because of the widespread connectivity that the internet grants, as well as the ease of expression of issues and concerns.

\section{Global Perspective}



\begin{thebibliography}{X}

  \bibitem{speedoflight-communication}
  https://www.extremetech.com/computing/151498-researchers-create-fiber-network-that-operates-at-99-7-speed-of-light-smashes-speed-and-latency-records

  \bibitem{world-population-online}
  https://www.invespcro.com/blog/world-population-online/

  \bibitem{physical-wellbeing-definition}
  https://www.uow.edu.au/student/wellbeing/UOW112647.html

  \bibitem{who-mental-wellbeing}
  http://www.who.int/features/factfiles/mental\_health/en/

  \bibitem{social-wellbeing-definition}
  https://www.uow.edu.au/student/wellbeing/UOW112638.html

  \bibitem{teens-social-media-brain}
  https://edition.cnn.com/2016/07/12/health/social-media-brain/index.html

	\bibitem{your-gadgets-are-robbing-you-of-your-sleep}
  https://timesofindia.indiatimes.com/home/sunday-times/your-gadgets-are-robbing-you-of-sleep/articleshow/64361117.cms

  \bibitem{india-internet-addicts}
  https://www.indiatoday.in/technology/news/story/india-has-most-internet-addicts-study-229324-2014-12-01

  \bibitem{digital-india-gov}
  https://www.mygov.in/group/digital-india/

  \bibitem{digital-india-thehindu}
  https://www.thehindu.com/opinion/columns/Chandrasekhar/economy-watch-column-by-cp-chandrasekhar-the-internet-in-digital-india/article7446778.ece

  \bibitem{digital-india-thehindu-2}
  https://www.thehindu.com/opinion/op-ed/the-grand-delusion-of-digital-india/article7727159.ece

\end{thebibliography}

\end{document}
