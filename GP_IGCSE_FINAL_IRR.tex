\documentclass[a4paper, 11pt]{article}

\usepackage[protrusion=true,expansion=true]{microtype} % Better typography
\usepackage{graphicx} % Required for including pictures
\usepackage{wrapfig} % Allows in-line images
\usepackage{hyperref}

\usepackage{mathpazo} % Use the Palatino font
\usepackage[T1]{fontenc} % Required for accented characters
\usepackage[superscript,biblabel]{cite}
\linespread{1.05} % Change line spacing here, Palatino benefits from a slight increase by default

\makeatletter

\renewcommand{\maketitle}{
\begin{flushright}
{\LARGE\@title}

\vspace{20pt}

{\large\@author}
\\\@date % Date

\vspace{40pt} % Some vertical space between the author block and abstract
\end{flushright}
}

\title{\textbf{How does the digital world affect the wellbeing of its users?}}

\author{\textsc{Vishnu Vivek Nittoor} \\
      Global Perspectives Individual Report\\
      {\textit{The International School Bangalore}}
      }

\date{\today}

\begin{document}
\maketitle

\vspace{30pt}

\section*{Introduction}

The digital world has evolved into a vast realm of possibilities in the past few decades. From dial-up modems to extremely fast fiberoptic cables, the availability, accessibility and increasing speed of the internet has facilitated this evolution. The internet is a tool with which users can communicate with others at staggering speeds, comparable to the speed of light, something that people nowadays take for granted more often than not.\cite{speedoflight-communication} The digital world has witnessed indiscriminate use in the past decades, with now 3.010 billion people having access to the internet.\cite{world-population-online}

For the purposes of this report, 'wellbeing' refers to an individual's state of being comfortable, healthy, or happy. This can also be further divided into physical, mental, and social wellbeing.\cite{physical-wellbeing-definition, who-mental-wellbeing, social-wellbeing-definition} This report will explore the ways in which the digital world affects these users' wellbeing.

\section{National Perspective}

\subsection{Social Media is Highly Addictive}
\label{national-highly-addictive}
Note that social media is a very large and significant section of the digital world, and that the effects of social media on its users' wellbeing also falls under the area of focus of this question. According to a study on the Indian youth, 73\% of respondents agreed that social media is highly addictive.\cite{indian-youth-study} To explain the cause of this, an article reveals that researchers at UCLA’s brain mapping centre found that when teenagers’ photos get lots of “likes” on social media apps, their brains’ reward centre is stimulated.\cite{teens-social-media-brain} This can cause social media to act like any other addictive substance, training the brain to use it because of the feeling associated in their brains with getting likes. Adolescents are more susceptible to this.

Related to this is the advanced algorithms of sites like YouTube. These are engineered in a way that takes into account the past videos watched by an individual, the interest exhibited in them, and uses extremely complex artificial intelligence and machine learning algorithms to determine what videos the user is likely to click on next.\cite{youtube-algorithm} This results in a user being constantly shown content which interests them, keeping them on the site for the longest amount of time so that ad revenue can be generated. This causes them to be addicted to the site.

The same study on the Indian youth shows that 76\% of respondents agreed that social networking leads to restlessness and sleeplessness.\cite{indian-youth-study} This can occur because of many reasons, where addiction is one of them. Because of addiction to social platforms, people cannot put down their phones because they will constantly be receiving notifications, messages and content from other people. The article says that “Researchers say we get addicted to these sounds over time”. This addiction to the sounds cause people to become alert whenever they get a notification - even when they are almost asleep. Since they have unknowlingly trained their brains to respond to the notification as soon as it arrives, they react to their first instinct and respond to it. Their response prompts the contact who is messaging them to respond due to the same phenomenon - and this continues. This can cause sleepless nights.

\subsection{Cyber-bullying}
Cyber-bullying or cyber harassment is a form of bullying or harassment using electronic means. Cyber-bullying is also known as online bullying. Based on a study conducted on the Indian youth regarding the impacts of social networking, 61\% of respondents agreed that social media gives rise to cyber-bullying. Furthermore, according to a 2014 study conducted by the Internet security company, McAfee, “Half of the youth in India have had some experience with cyber-bullying”.

This can cause victims to suffer a considerable drop in self-esteem, while being a constant source of distress in their lives. This causes them to spend time in withdrawal from family members and friends, and experience anxiety regarding other individuals checking their mobile phone, laptop, etc, and can even lead to victims committing suicide in extreme cases. Cyber-bullying extremely negatively affects an individual’s social well-being. If this is a consequence of social media, then data leads us to believe that Indian users are prone to be very negatively affected by exposure to the digital world.

\subsection{Digital India}
\label{digital-india}
On the other hand, the Indian government has launched a campaign called 'Digital India', which aims to "empower every citizen with access to digital services, knowledge and information", because "Digital technologies are being increasingly used by us in everyday lives from retail stores to government offices. They help us to connect with each other and also to share information on issues and concerns faced by us. In some cases they also enable resolution of those issues in near real time".\cite{digital-india-gov} It includes initiatives to reach internet to rural areas and improve accessibility for all Indian citizens. Though many groups of people express skepticism, this reflects the Indian government's belief that access to the digital world improves the wellbeing of India's citizens, because of the widespread connectivity that the internet grants, as well as the ease of expression of issues and concerns. \cite{digital-india-thehindu, digital-india-thehindu-2}

\section{Global Perspective}

\subsection{Online Shaming}

Online shaming is a phenomenon observed on the internet where targets are publicly humiliated because of an action they carried out which violated social norms or values. According to Jon Ronson, the author of the book "So You've Been Publicly Shamed", social media like Twitter gives voice to the voiceless, allowing them to speak up and hit back at perceived injustice. \cite{jon-ronson-book, jon-ronson-ted-talk} For example, Justine Sacco, a woman with 170 followers on Twitter posted an offensive tweet before she got into an airplane for an eleven hour journey. After she had landed, she found that she was the number one trending topic on twitter, that she had been getting condemning and denouncing tweets targeted at her, and due to social outrage, had even lost her PR job.

Benjamin Rush, who signed the USA's Declaration of Independence argued that public shaming "is universally acknowledged to be a worse punishment than death". Online shaming is considered to be the same thing, except done through social media. Ronson expresses his opinion that "social justice" carried out through online platforms has been taken too far. In his TED talk, he says: "The people I met were mangled. They talked to me about depression, and anxiety and insomnia and suicidal thoughts. One woman I talked to, who also told a joke that landed badly, she stayed home for a year and a half".\cite{jon-ronson-ted-talk} According to a UK article on online shaming, uploading an accusatory video or tweet online takes just a few seconds, but can lead to very deep ramifications for individuals that can last a lifetime.\cite{telegraph-online-shaming} According to another book on online shaming, strong episodes of negative exposure on the internet can cause very serious consequences like loss of a job, social ostracism, and even post-traumatic stress disorder.\cite{shame-nation-book}

While acknowledging this, an article on Bloomberg magazine also reveals how online shaming can lead to good societal outcomes if done in moderation.\cite{bloomberg-online-shaming} It says that "shaming can also be good for society, because it allows us to hold people and organizations responsible for bad behavior". For example, Daraprim, a generic medication used to treat infections in AIDS patients was bought by Martin Shkreli's company, which then raised the price from \$13.50 per pill to \$750, an increase of over 5000\%.\cite{washington-martin} This sparked overwhelming internet outrage, where even Hillary Clinton made a statement calling this action "outrageous price-gouging". After suffering a great deal of internet humiliation, Shkreli said that Daraprim will no longer retail at \$750 as a response to the "the anger that was felt by people". This is way online shaming shows its power to affect positive change in the wellbeing of people.

\section{Personal Perspective}

I am an active and frequent user of technology and the digital world. I connect with my school friends through platforms like WhatsApp and Quora. I use the internet to interact with the world around me, for my education, research, hobbies, media consumption, playing video games, and to connect with my friends and more. The internet has changed my life in many ways, especially the way I interact with other people in my life, and augmented other aspects of my life like shopping and news. It has definitely made our lives easier, just like any kind of technological progress does.

A very significant way I spend time on the internet is doing online courses. Online platforms like edX allow users to take courses from various universities, and I have been very actively taking such courses for the past two years. Over the span of two years, I have finished nine courses from many different prestigious universities, which are opportunities that I would not have access to without the digital world. In fact, it has helped me discover my inner passion for mathematics, philosophy, and computer science. I have been able to gain a profound understanding of such subjects, which helped me find clarity in what I would like to pursue in the future. I feel that this has greatly and positively influenced my personal wellbeing and contentment, since I have discovered my hidden interests through the internet.

A major chunk of my time is spent programming. I find this extremely enjoyable, as I had discovered through my experience of online courses. However, there are a lot of difficulties I have encountered while coding, which is normal for any amateur programmer. As I live in a dormitory, I do not have easy access to an expert who can look at my code in person. So, I created an account on \emph{stackoverflow}, a digital community where over 50 million users learn, interact and share their knowledge in computer science and programming through a question-answer system. I found this instrumental to my growth as a programmer, as I was able to stand on the shoulders of millions of others like me who had the same potential questions. If my question was previously unasked, I could put it up on the site so that someone could answer it - and I got results, which worked, within minutes. I feel that the digital world has proven to be very beneficial because of the ability to form online communities of users who have a common interest and purpose, in order to help each other learn and improve by sharing knowledge. This has saved me countless hours of frustration and helped me fix my problems almost instantly.

However, despite my daily use of the internet, I do not have the perspective that the digital world only helps our wellbeing and doesn't harm us in any way. Most of my concerns with the digital world are with social media. I find that I agree with the perspective that social media is highly addictive, expressed in section \ref{national-highly-addictive}. I live in a dormitory, where distracting sites are blocked using the school firewall. At home, I find myself constantly checking my phone for notifications, being distracted by sites like YouTube and Netflix when I have to do work, and having to resort to measures like locking my phone in another room or using a website blocker. Seeing as I do not fall prey to such impulses in the dormitory, I am led to believe that this is because of the unrestricted access to the digital world that I have at home.

Social media applications have made it so easy to communicate with others. Because of the availability and accessibility that the digital world endorses, it now feels necessary to check an incoming notification when it arrives. This is also the same phenomenon described in section \ref{national-highly-addictive}, and I personally agree with it. Because it is so easy to quickly reply to the notification as soon as it arrives, it is hard to not give into the apparent harmlessness of checking it. The need to be available all the time bothers me, so I keep my phone close to me at all times when I am at home. In the dormitory, the inability to be available leads to more peace of mind.

After considering how the digital world improves my life in so many ways yet distracts me, I end up with a complicated personal perspective which is a mix of what is expressed in sections \ref{national-highly-addictive} and \ref{digital-india}.

\begin{thebibliography}{X}

  \bibitem{speedoflight-communication}
  Anthony, Sebastian. “Researchers Create Fiber Network That Operates at 99.7\% Speed of Light, Smashes Speed and Latency Records.” ExtremeTech, 27 Mar. 2013, \href{https://www.extremetech.com/computing/151498-researchers-create-fiber-network-that-operates-at-99-7-speed-of-light-smashes-speed-and-latency-records}{https://www.extremetech.com/computing/151498-researchers-create-fiber-network-that-operates-at-99-7-speed-of-light-smashes-speed-and-latency-records}.

  \bibitem{world-population-online}
  “Physical Wellbeing.” University of Wollongong Australia, 26 Feb. 2015, \href{https://www.uow.edu.au/student/wellbeing/UOW112647.html}{https://www.uow.edu.au/student/wellbeing/UOW112647.html}.

  \bibitem{physical-wellbeing-definition}
  Saleh, Khalid. “How Much Of The World Population Is Online – Statistics And Trends.” Invespcro, \href{https://www.invespcro.com/blog/world-population-online/}{https://www.invespcro.com/blog/world-population-online/}.

  \bibitem{who-mental-wellbeing}
  “Mental Health: a State of Well-Being.” World Health Organization, World Health Organization, 15 Aug. 2014, \href{https://www.who.int/features/factfiles/mental\_health/en/}{https://www.who.int/features/factfiles/mental\_health/en/}.

  \bibitem{social-wellbeing-definition}
  “Social Wellbeing.” University of Wollongong Australia, 26 Feb. 2015, \href{https://www.uow.edu.au/student/wellbeing/UOW112638.html}{https://www.uow.edu.au/student/wellbeing/UOW112638.html}.

  \bibitem{teens-social-media-brain}
  East, Susie. “How Does Social Media Affect Your Brain.” CNN, Cable News Network, 1 Aug. 2016, \href{https://edition.cnn.com/2016/07/12/health/social-media-brain/index.html}{https://edition.cnn.com/2016/07/12/health/social-media-brain/index.html}.

  \bibitem{indian-youth-study}
  Bhardwaj, Akashdeep and Goundar, Sam and Avasthi, Vinay. (2017). Impact of Social Networking on Indian Youth-A Survey. International Journal of Electronics and Telecommunications. 7. 41-51. 10.6636/IJEIE.201709.7(1).05). \href{https://www.researchgate.net/publication/322538324\_Impact\_of\_Social\_Networking\_on\_Indian\_Youth-A\_Survey}{https://www.researchgate.net/publication/322538324\_Impact\_of\_Social\_\\Networking\_on\_Indian\_Youth-A\_Survey}

	\bibitem{your-gadgets-are-robbing-you-of-your-sleep}
  https://timesofindia.indiatimes.com/home/sunday-times/your-gadgets-are-robbing-you-of-sleep/articleshow/64361117.cms

  \bibitem{india-internet-addicts}
  https://www.indiatoday.in/technology/news/story/india-has-most-internet-addicts-study-229324-2014-12-01

  \bibitem{digital-india-gov}
  https://www.mygov.in/group/digital-india/

  \bibitem{digital-india-thehindu}
  https://www.thehindu.com/opinion/columns/Chandrasekhar/economy-watch-column-by-cp-chandrasekhar-the-internet-in-digital-india/article7446778.ece

  \bibitem{digital-india-thehindu-2}
  https://www.thehindu.com/opinion/op-ed/the-grand-delusion-of-digital-india/article7727159.ece

  \bibitem{youtube-algorithm}
  Blattmann, Josefina. “The Secret behind YouTube's Algorithm Is You – Elevate by Lateral View – Medium.” Medium, Medium, 30 Nov. 2017, \href{https://medium.com/elevate-by-lateral-view/the-secret-behind-youtube-algorithm-is-you-8fae5bfdf8ca}{https://medium.com/elevate-by-lateral-view/the-secret-behind-youtube-algorithm-is-you-8fae5bfdf8ca}

  \bibitem{jon-ronson-book}
  Ronson, Jon. So You've Been Publicly Shamed. Picador, 2016.

  \bibitem{jon-ronson-ted-talk}
  Ronson, Jon. “When Online Shaming Goes Too Far.” TEDGlobalLondon. TEDGlobalLondon, June 2015.
  \href{https://www.ted.com/talks/jon\_ronson\_what\_happens\_when\_online\_shaming\_spirals\_out\_of\_control?language=en}{https://www.ted.com/talks/jon\_ronson\_what\_happens\_when\_online\_shaming\_spirals\_out\_of\_control?language=en}

  \bibitem{telegraph-online-shaming}
  https://www.telegraph.co.uk/news/2018/06/25/online-shaming-dangerous-rise-internet-pitchfork-mob/

  \bibitem{bloomberg-online-shaming}
  https://www.bloomberg.com/view/articles/2017-12-04/social-media-shaming-is-good-in-moderation

  \bibitem{shame-nation-book}
  Scheff, Sue, et al. Shame Nation: the Global Epidemic of Online Hate. Sourcebooks, Inc., 2018.

  \bibitem{washington-martin}
  https://www.washingtonpost.com/news/the-intersect/wp/2015/09/23/the-successful-internet-shaming-of-pharma-bro-martin-shkreli/?utm\_term=.629553b567bf

\end{thebibliography}

\end{document}
