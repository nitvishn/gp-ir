\documentclass[a4paper, 11pt]{article}

\usepackage[protrusion=true,expansion=true]{microtype} % Better typography
\usepackage{graphicx} % Required for including pictures
\usepackage{wrapfig} % Allows in-line images
\usepackage{hyperref}

\usepackage{mathpazo} % Use the Palatino font
\usepackage[T1]{fontenc} % Required for accented characters
\usepackage[superscript,biblabel]{cite}
\linespread{1.05} % Change line spacing here, Palatino benefits from a slight increase by default

\makeatletter

\renewcommand{\maketitle}{
\begin{flushright}
{\LARGE\@title}

\vspace{20pt}

{\large\@author}
\\\@date % Date

\vspace{40pt} % Some vertical space between the author block and abstract
\end{flushright}
}

\title{\textbf{How does the digital world affect the wellbeing of its users?}}

\author{\textsc{Vishnu Vivek Nittoor} \\
      Global Perspectives Individual Report\\
      {\textit{The International School Bangalore}}
      }

\date{\today}

\begin{document}
\maketitle

\vspace{30pt}

\section*{Introduction}

The digital world has evolved into a vast realm of possibilities in the past few decades. From dial-up modems to extremely fast fiberoptic cables, the availability, accessibility and increasing speed of the internet has facilitated this evolution. The internet is a tool with which users can communicate with others at staggering speeds, something that people nowadays take for granted more often than not.\cite{speedoflight-communication} The digital world has witnessed indiscriminate use in the past decades, with now 3.010 billion people having access to the internet.\cite{world-population-online}

For the purposes of this report, 'wellbeing' refers to an individual's state of being comfortable, healthy, or happy. This can also be further divided into physical, mental, and social wellbeing.\cite{physical-wellbeing-definition, who-mental-wellbeing, social-wellbeing-definition}

\section{Issue: Connectivity and Addiction}

    Note that social media is a very large and significant section of the digital world, and that the effects of social media on its users' wellbeing also falls under the area of focus of this question.

    The advanced algorithms of sites like YouTube are engineered in a way that takes into account the past videos watched by an individual, the interest exhibited in them, and uses extremely complex artificial intelligence and machine learning algorithms to determine what videos the user is likely to click on next.\cite{youtube-algorithm} This results in a user being constantly shown content which interests them, keeping them on the site for the longest amount of time so that ad revenue can be generated. This causes them to be addicted to the site.

    \subsection{National Perspective}
      According to a study on the Indian youth, 73\% of respondents agreed that social media is highly addictive.\cite{indian-youth-study} According to this article, when a user constantly receives social media messages, they get addicted to the sounds over time. \cite{your-gadgets-are-robbing-you-of-your-sleep} This can even cause them to lose out on sleep, as confirmed by the same study on the Indian youth. 76\% of respondents agree that social networking leads to restlessness and sleeplessness. Related to this is "Instant Gratification", a mindset where people need their wants fulfilled as soon as possible. \cite{instant-grat, instant-grat-india}

      However, the government believes in digital connectivity. The Indian government has launched a campaign called 'Digital India', which aims to "empower every citizen with access to digital services, knowledge and information", because digital technologies "help us to connect with each other and also to share information on issues and concerns faced by us. In some cases they also enable resolution of those issues in near real time".\cite{digital-india-gov} For example, citizens will be able to have access to government services online. They can now access government documents and certificates through the cloud, and even submit comments on the resolution of issues and problems. It includes initiatives to provide high speed internet in rural areas. Though many groups of people express skepticism, this reflects the Indian government's belief that access to the digital world improves the wellbeing of India's citizens, because of the widespread connectivity that the internet grants, as well as the ease of expression of issues and concerns. \cite{digital-india-thehindu, digital-india-thehindu-2}

    \subsection{Global Perspective}
      According to research, connectivity can also have its adverse effects. The word "FOMO" was added to the Oxford English Dictionary in 2013. It is an abbreviation of the words "Fear of Missing Out". This is a phenomenon that was first discovered by Dr. Dan Herman, which is social media users' fear of being absent when others might be having rewarding experiences.\cite{wikipedia-fomo} This anxiety can be seen when a user wants to be continually connected to what others are doing by checking social media persistently, which is linked to feelings of dissatisfaction and disconnection of people from their own lives.\cite{fomo-science}

      Researchers at UCLA’s brain mapping centre found that when teenagers’ photos get lots of “likes” on social media apps, their brains’ reward centre is stimulated.\cite{teens-social-media-brain} This can cause social media to act like any other addictive substance, training the brain to use it because of the feeling associated in their brains with getting likes. Adolescents are more susceptible to this.

      According to this article, a study by the University of Michigan on Facebook users shows that it made them sadder and less satisfied.\cite{facebook-sad} Researchers suspect that this is because users tend to compare their lives to what they see online - and this results in unhappiness. Users only project their best and happiest moments online, and comparing this to the rest of someone's life results in unhappiness and inadequacy.

      This article expresses the viewpoint that online activism is having a positive effect in the real world.\cite{nytimes-connectedness} Campaigns like the ALS Ice Bucket Challenge have raised up to \$115,000,000 by propagating via the digital world, and was also responsible for a breakthrough in ALS research because of its funding. The UN is also making use of the internet to raise awareness for its Sustainable Development Goals.\cite{als-wiki, als-challenge} This displays the internet's ability to bring about positive change in wellbeing.

  \subsection{Course of action}

    In my opinion, educating users about the dangerous aspects of increasing online connectivity can help their chances of being less affected by these dangers. In fact, Google is developing a new app for Android called "Digital Wellbeing", to help users become more aware of internet overuse, and connect to the digital world in a safer, more controlled way to improve their daily wellbeing.

\section{Issue: Online Harassment}

  Online harassment or cyber-bullying is a form of bullying or harassment using electronic means. Cyber-bullying is also known as online bullying. A major part of this is online shaming. This is a phenomenon observed on the internet where targets are publicly humiliated because of an action they carried out which violated social norms or values. According to Jon Ronson, the author of the book "So You've Been Publicly Shamed", social media like Twitter gives voice to the voiceless, allowing them to speak up and hit back at perceived injustice. \cite{jon-ronson-book, jon-ronson-ted-talk}

  \subsection{National Perspective}

    Based on a study conducted on the Indian youth regarding the impacts of social networking, 61\% of respondents agreed that social media gives rise to cyber-bullying.\cite{indian-youth-study} Furthermore, according to a 2014 study conducted by the Internet security company, McAfee, “Half of the youth in India have had some experience with cyber-bullying”.
    If this is a consequence of social media, then data leads us to believe that Indian users are prone to be very negatively affected by exposure to the digital world.

    This article from the Indian newspaper, Hindustan Times, shows how Indians are using social media to expose sexual harassment.\cite{hindustan-times-shaming} It says that social media has proved its worth "empowering" women across the country, as many cases of sexual harassment have been successfully reported through the use of social media like Twitter and Facebook. In addition, cyber crime expert Priya Bannerjee also highlights the fact that the police now have an online presence, and will consider such social media posts as FIRs (First Information Reports).

    However, there does exist the possibility of an accusatory social media post being made while the accused is actually innocent. This viewpoint is expressed in the same article. \cite{hindustan-times-shaming} Even if the accused is proven to be innocent in a court of law, this accusation will be made permanent because it has been posted on the internet, leading to potential extended public shaming of the accused due to misunderstandings. Permanent online association with that crime can be a red flag in the eyes of employers, even if the accused is innocent.

  \subsection{Global Perspective}
    Justine Sacco, a woman with 170 followers on Twitter posted an offensive tweet before she got into an airplane for an eleven hour journey. After she had landed, she found that she was the number one trending topic on twitter, that she had been getting condemning and denouncing tweets targeted at her, and due to social outrage, had even lost her PR job.

    Benjamin Rush, who signed the USA's Declaration of Independence argued that public shaming "is universally acknowledged to be a worse punishment than death". Online shaming is considered to be the same thing, except done through social media. Jon Ronson expresses his opinion that "social justice" carried out through online platforms has been taken too far. In his TED talk, he says: "The people I met were mangled. They talked to me about depression, and anxiety and insomnia and suicidal thoughts. One woman I talked to, who also told a joke that landed badly, she stayed home for a year and a half".\cite{jon-ronson-ted-talk} According to a book on online shaming, strong episodes of negative exposure on the internet can cause very serious consequences like loss of a job, social ostracism, and even post-traumatic stress disorder.\cite{shame-nation-book}

  \subsection{Course of action}

    An article on Bloomberg magazine also reveals how online shaming can lead to good societal outcomes if done in moderation.\cite{bloomberg-online-shaming} It says that "shaming can also be good for society, because it allows us to hold people and organizations responsible for bad behavior". For example, Daraprim, a generic medication used to treat infections in AIDS patients was bought by Martin Shkreli's company, which then raised the price from \$13.50 per pill to \$750, an increase of over 5000\%.\cite{washington-martin} This sparked overwhelming internet outrage, which consequently led to Shkreli saying that Daraprim will no longer retail at \$750 as a response to the "the anger that was felt by people". This is a display of the positive power of online shaming to affect positive change in the wellbeing of people.
    The potential power of online shaming must be brought to notice. However, users should be made aware about its consequences if done in extremity.


\section*{Personal Perspective}

  I am an active and frequent user of technology and the digital world. I use the internet to interact with the world around me, for my education, research, hobbies, media consumption, playing video games, and to connect with my friends and more. The internet has changed my life in many ways, especially the way I interact with other people in my life, and augmented other aspects of my life. It has definitely made our lives easier, just like any kind of technological progress does.

  A very significant way I spend time on the internet is doing online courses on online platforms like edX. This allows me to take otherwise unaccessible courses from various prestigious universities around the world. In fact, it has helped me discover and gain a profound understanding in my inner passion for mathematics, philosophy, and computer science. I feel that this has greatly and positively influenced my personal wellbeing and contentment, since I now see what I would like to do in the future with greater clarity.

  A major chunk of my spare time is spent programming. I find this extremely enjoyable, as I had discovered through my experience of online courses. Whenever I encounter a difficulty while programming, I consult an online community called \emph{stackoverflow}, which is a platform where over 50 million users like me learn and share information about programming. I found this instrumental to my growth as a programmer. This saved me many hours of frustration and helped me fix my problems almost instantly. This is one of countless examples where the digital world has positively revolutionised how I do things.

  Most of my concerns with the digital world are with social media. I find that I agree that social media is highly addictive. I live in a dormitory, where distracting sites are blocked using the school firewall. At home, I find myself distracted by my phone as well as sites like YouTube and Netflix, and having to resort to measures like locking my phone in another room or using a website blocker. Seeing as I do not fall prey to such impulses in the dormitory, I am led to believe that this is because of the unrestricted access to the digital world that I have at home. I feel that the inability to be available, paradoxically, leads to more peace of mind.

\subsection{Conclusion}

  Before beginning my research, I was biased to the viewpoint that the digital world positively affects general wellbeing, since it was such a huge part of my life. Though I found reasons to believe that the digital world also can have very unfavourable consequences on its users, I also found reasons to believe that there is some truth in my presumption. However, I had not fully realised how it was negatively affecting my life at home until I began researching about its effects. Now, I am more aware of its dangers, while still being reassured that wise use of the digital world can greatly influence my wellbeing.

\begin{thebibliography}{X}

  \bibitem{speedoflight-communication}
  Anthony, Sebastian. “Researchers Create Fiber Network That Operates at 99.7\% Speed of Light, Smashes Speed and Latency Records.” ExtremeTech, 27 Mar. 2013, \href{https://www.extremetech.com/computing/151498-researchers-create-fiber-network-that-operates-at-99-7-speed-of-light-smashes-speed-and-latency-records}{https://www.extremetech.com/computing/151498-researchers-create-fiber-network-that-operates-at-99-7-speed-of-light-smashes-speed-and-latency-records}.

  \bibitem{world-population-online}
  “Physical Wellbeing.” University of Wollongong Australia, 26 Feb. 2015, \href{https://www.uow.edu.au/student/wellbeing/UOW112647.html}{https://www.uow.edu.au/student/wellbeing/UOW112647.html}.

  \bibitem{physical-wellbeing-definition}
  Saleh, Khalid. “How Much Of The World Population Is Online – Statistics And Trends.” Invespcro, \href{https://www.invespcro.com/blog/world-population-online/}{https://www.invespcro.com/blog/world-population-online/}.

  \bibitem{who-mental-wellbeing}
  “Mental Health: a State of Well-Being.” World Health Organization, World Health Organization, 15 Aug. 2014, \href{https://www.who.int/features/factfiles/mental\_health/en/}{https://www.who.int/features/factfiles/mental\_health/en/}.

  \bibitem{social-wellbeing-definition}
  “Social Wellbeing.” University of Wollongong Australia, 26 Feb. 2015, \href{https://www.uow.edu.au/student/wellbeing/UOW112638.html}{https://www.uow.edu.au/student/wellbeing/UOW112638.html}.

  \bibitem{teens-social-media-brain}
  East, Susie. “How Does Social Media Affect Your Brain.” CNN, Cable News Network, 1 Aug. 2016, \href{https://edition.cnn.com/2016/07/12/health/social-media-brain/index.html}{https://edition.cnn.com/2016/07/12/health/social-media-brain/index.html}.

  \bibitem{indian-youth-study}
  Bhardwaj, Akashdeep and Goundar, Sam and Avasthi, Vinay. (2017). Impact of Social Networking on Indian Youth-A Survey. International Journal of Electronics and Telecommunications. 7. 41-51. 10.6636/IJEIE.201709.7(1).05). \href{https://www.researchgate.net/publication/322538324\_Impact\_of\_Social\_Networking\_on\_Indian\_Youth-A\_Survey}{https://www.researchgate.net/publication/322538324\_Impact\_of\_Social\_\\Networking\_on\_Indian\_Youth-A\_Survey}

	\bibitem{your-gadgets-are-robbing-you-of-your-sleep}
  “Your Gadgets Are Robbing You of Sleep - Times of India.” The Times of India, 1 June 2018, \href{https://timesofindia.indiatimes.com/home/sunday-times/your-gadgets-are-robbing-you-of-sleep/articleshow/64361117.cms}{https://timesofindia.indiatimes.com/home/sunday-times/your-gadgets-are-robbing-you-of-sleep/articleshow/64361117.cms}.

  \bibitem{digital-india-gov}
  “Digital India.” Government of India, 6 Apr. 2015, \href{https://www.mygov.in/group/digital-india/}{https://www.mygov.in/group/digital-india/}.

  \bibitem{digital-india-thehindu}
  Chandrasekhar, C. P. “The Internet in ‘Digital India.’” The Hindu, The Hindu, 21 July 2015, \href{https://www.thehindu.com/opinion/columns/Chandrasekhar/economy-watch-column-by-cp-chandrasekhar-the-internet-in-digital-india/article7446778.ece}{https://www.thehindu.com/opinion/columns/Chandrasekhar/economy-watch-column-by-cp-chandrasekhar-the-internet-in-digital-india/article7446778.ece}.

  \bibitem{digital-india-thehindu-2}
  Mannathukkaren, Nissim. “The Grand Delusion of Digital India.” The Hindu, The Hindu, 2 Sept. 2016, \href{https://www.thehindu.com/opinion/op-ed/the-grand-delusion-of-digital-india/article7727159.ece}{https://www.thehindu.com/opinion/op-ed/the-grand-delusion-of-digital-india/article7727159.ece}.

  \bibitem{youtube-algorithm}
  Blattmann, Josefina. “The Secret behind YouTube's Algorithm Is You – Elevate by Lateral View – Medium.” Medium, Medium, 30 Nov. 2017, \href{https://medium.com/elevate-by-lateral-view/the-secret-behind-youtube-algorithm-is-you-8fae5bfdf8ca}{https://medium.com/elevate-by-lateral-view/the-secret-behind-youtube-algorithm-is-you-8fae5bfdf8ca}

  \bibitem{jon-ronson-book}
  Ronson, Jon. So You've Been Publicly Shamed. Picador, 2016.

  \bibitem{jon-ronson-ted-talk}
  Ronson, Jon. “When Online Shaming Goes Too Far.” TEDGlobalLondon. TEDGlobalLondon, June 2015.
  \href{https://www.ted.com/talks/jon\_ronson\_what\_happens\_when\_online\_shaming\_spirals\_out\_of\_control?language=en}{https://www.ted.com/talks/jon\_ronson\_what\_happens\_when\_online\_shaming\_spirals\_out\_of\_control?language=en}

  \bibitem{telegraph-online-shaming}
  Molloy, Mark. “Online Shaming: The Dangerous Rise of the Internet Pitchfork Mob   .” The Telegraph, Telegraph Media Group, 25 June 2018, \href{https://www.telegraph.co.uk/news/2018/06/25/online-shaming-dangerous-rise-internet-pitchfork-mob/}{https://www.telegraph.co.uk/news/2018/06/25/online-shaming-dangerous-rise-internet-pitchfork-mob/}.

  \bibitem{bloomberg-online-shaming}
  Alaimo, Kara. “Social-Media Shaming Is Good (in Moderation).” Bloomberg.com, Bloomberg, 5 Dec. 2017, \href{https://www.bloomberg.com/view/articles/2017-12-04/social-media-shaming-is-good-in-moderation}{https://www.bloomberg.com/view/articles/2017-12-04/social-media-shaming-is-good-in-moderation}.

  \bibitem{shame-nation-book}
  Scheff, Sue, et al. Shame Nation: the Global Epidemic of Online Hate. Sourcebooks, Inc., 2018.

  \bibitem{washington-martin}
  Dewey, Caitlin. “The Successful Internet Shaming of 'Pharma Bro' Martin Shkreli.” The Washington Post, WP Company, 23 Sept. 2015, \href{https://www.washingtonpost.com/news/the-intersect/wp/2015/09/23/the-successful-internet-shaming-of-pharma-bro-martin-shkreli/}{https://www.washingtonpost.com/news/the-intersect/wp/2015/09/23/the-successful-internet-shaming-of-pharma-bro-martin-shkreli/}.

  \bibitem{wikipedia-fomo}
  “Fear of Missing Out.” Wikipedia, Wikimedia Foundation, 22 Sept. 2018, \href{https://en.wikipedia.org/wiki/Fear\_of\_missing\_out}{https://en.wikipedia.org/wiki/Fear\_of\_missing\_out}.

  \bibitem{fomo-science}
  "FOMO: It's Your Life You're Missing out On.” ScienceDaily, ScienceDaily, 30 Mar. 2016, \href{www.sciencedaily.com/releases/2016/03/160330135623.htm}{www.sciencedaily.com/releases/2016/03/160330135623.htm}.

  \bibitem{hindustan-times-shaming}
  Bawa, Jyoti Sharma. “Woman's Always Right? Naming, Shaming 'Molestors' on Social Media.” Hindustan Times, 25 Aug. 2015, \href{https://www.hindustantimes.com/analysis/woman-s-always-right-naming-shaming-molestors-on-social-media/story-c19z6LkyUFxGrxvoWNYwyO.html}{https://www.hindustantimes.com/analysis/woman-s-always-right-naming-shaming-molestors-on-social-media/story-c19z6LkyUFxGrxvoWNYwyO.html}.

  \bibitem{nytimes-connectedness}
  Csorba, Emerson, and Noa Gafni Slaney. “Is Digital Connectedness Good or Bad for People?” The New York Times, The New York Times, 28 Nov. 2016, \href{https://www.nytimes.com/roomfordebate/2016/11/28/is-digital-connectedness-good-or-bad-people}{https://www.nytimes.com/roomfordebate/2016/11/28/is-digital-connectedness-good-or-bad-people}.

  \bibitem{als-wiki}
  “Amyotrophic Lateral Sclerosis.” Wikipedia, Wikimedia Foundation, 1 Oct. 2018, \href{https://en.wikipedia.org/wiki/Amyotrophic\_lateral\_sclerosis}{https://en.wikipedia.org/wiki/Amyotrophic\_lateral\_sclerosis}.

  \bibitem{als-challenge}
  “Ice Bucket Challenge.” Wikipedia, Wikimedia Foundation, 13 Sept. 2018, \href{https://en.wikipedia.org/wiki/Ice\_Bucket\_Challenge}{https://en.wikipedia.org/wiki/Ice\_Bucket\_Challenge}.

  \bibitem{facebook-sad}
  Hu, Elise. “Facebook Makes Us Sadder And Less Satisfied, Study Finds.” NPR, NPR, 20 Aug. 2013, \href{https://www.npr.org/sections/alltechconsidered/2013/08/19/213568763/researchers-facebook-makes-us-sadder-and-less-satisfied}{https://www.npr.org/sections/alltechconsidered/2013/08/19/213568763/researchers-facebook-makes-us-sadder-and-less-satisfied}.

  \bibitem{instant-grat}
  Tulipano, Rachael. “Brief Happiness: The Truth Behind Why We Want Instant Gratification.” Elite Daily, Elite Daily, 20 June 2018, \href{https://www.elitedaily.com/social-news/instant-gratification/1157913}{https://www.elitedaily.com/social-news/instant-gratification/1157913}.

  \bibitem{instant-grat-india}
  Nangia, Vinita. “An Age of Instant Gratification!” Times of India Blog, 3 Apr. 2016, \href{https://blogs.timesofindia.indiatimes.com/O-zone/an-age-of-instant-gratification/}{https://blogs.timesofindia.indiatimes.com/O-zone/an-age-of-instant-gratification/}.


\end{thebibliography}

\end{document}
